\documentclass[twoside,letterpaper,12pt]{report}
\usepackage[spanish,backgroundcolor=green,textsize=small]{todonotes}
\usepackage[top=3cm,bottom=3cm,left=3cm,right=3cm]{geometry} 
\usepackage[utf8x]{inputenc}
\usepackage[spanish]{babel}

\usepackage{amsmath}
\usepackage{amssymb}
\usepackage{times}
\usepackage{color}
\usepackage{graphicx}
\usepackage{framed}
\usepackage{wrapfig}
\usepackage{hyperref}
\usepackage{parskip}
\usepackage{float}
\usepackage[section]{placeins} 
%\usepackage[table]{xcolor}
%\usepackage{plain} % no se de que es
%\usepackage{a4wide} % para el ancho de pagina
%\usepackage{multirow} % tablas con multicolumna

\setlength{\parindent}{15pt}

\hypersetup{
colorlinks=true,
linkcolor=black,          % color of internal links (change box color with linkbordercolor)
citecolor=black,        % color of links to bibliography
filecolor=black,      % color of file links
urlcolor=black           % color of external links
}

\usepackage{listings}
\usepackage{courier}
\lstset{
basicstyle=\footnotesize\ttfamily, % Standardschrift
numbers=left,               % Ort der Zeilennummern
numberstyle=\tiny,          % Stil der Zeilennummern
%stepnumber=2,               % Abstand zwischen den Zeilennummern
numbersep=5pt,              % Abstand der Nummern zum Text
tabsize=2,                  % Groesse von Tabs
extendedchars=true,         %
breaklines=true,            % Zeilen werden Umgebrochen
keywordstyle=\color{red},
frame=b,         
%        keywordstyle=[1]\textbf,    % Stil der Keywords
%        keywordstyle=[2]\textbf,    %
%        keywordstyle=[3]\textbf,    %
%        keywordstyle=[4]\textbf,   \sqrt{\sqrt{}} %
stringstyle=\color{white}\ttfamily, % Farbe der String
showspaces=false,           % Leerzeichen anzeigen ?
showtabs=false,             % Tabs anzeigen ?
xleftmargin=17pt,
framexleftmargin=17pt,
framexrightmargin=5pt,
framexbottommargin=4pt,
%backgroundcolor=\color{lightgray},
showstringspaces=false      % Leerzeichen in Strings anzeigen ?        
}
\lstloadlanguages{% Check Dokumentation for further languages ...
%[Visual]Basic
%Pascal
%C
sh,
C++
%XML
%HTML
%Java
}

\title{Parcial 1\\[1cm]
Fundamentos de Biología Computacional}

\author{
	Sergio Andrés Monsalve Castañeda \\
	Código: 201410029114\\[0.5cm]	
	\includegraphics[width=0.25\textwidth]{logo_eafit} \\[1cm]
	\textbf{Profesores:} \\
	Javier Correa \\
	Alejandro Rodríguez \\[1cm]
	Departamento de Informática y Sistemas \\
	Universidad EAFIT\\[0.5cm] 
	Copyright \copyright \hspace{3pt}
}

\date{\today}

\begin{document}
\maketitle

\begin{enumerate}
	\item  Los siguientes son secuencias de anticodones que participaron en la síntesis de una proteína: \textbf{ UAC UAA	CGC	CCC	AGA	UCA}. Derive:

\begin{itemize}
	\item La cadena de ARNm: AUG AUU GCG GGG UCU AGU
	\item La cadena de ADN que sirvió de molde: En el sentido 3' 5' la cadena molde es: \textbf{TAC TAA CGC CCC AGA TCA}
	\item Teniendo en cuenta en la siguiente tabla de codones, traduzca la secuencia de aminoácidos correspondiente: \text{Met Ile Ala Gly Ser Ser}
\end{itemize}

	\item ¿Cuál es la Relación entre un alineamiento de secuencias y evolución?

	\textbf{Respuesta:}  Con el alineamiento de secuencias se puede encontrar un porcentaje de identidad que de muestra de la similitud entre organismos. 

	A medida que organismos con ancestros en común evolucionan, van acumulando mutaciones en el genoma, las  cuales derivan en especies diferentes. De acuerdo a la cantidad de diferencias encontradas en el alineamiento, se puede deducir la cercanía o no entre especies. 
 
	Cuando el porcentaje de identidad sea igual al 100\% existe una alta probabilidad de que las secuencias correspondan al mismo organismo.

	\item Según el modelo estadístico de Karlin-Altschul(1990), el cual hace referencia a la estimación del puntaje más alto(HSP) en un alineamiento de dos secuencias. ¿Cómo es argumentado este modelo y que aplicabilidad tiene en la búsqueda de homología de secuencias hoy en día?. Explique brevemente el modelo.



	\item Represente como sería una matriz de alineamiento global para las siguientes dos proteínas. Asuma los valores estándar de puntuación que hay reportado en la literatura para aminoácidos.	

	\begin{enumerate}
		\item SPAALKALAEAAGS	
 		\item SGAALKALAEAAPS	 
	\end{enumerate}

	\item Consulte qué tipo de algoritmos de alineamiento utilizan los programas de FASTA, CAP3, Bowtie. Compare los tres en términos de velocidad, capacidad de análisis sus principales limitaciones.

	\begin{tabular}{|r|c|c|c|c|}
		\hline 
		Programa & Algoritmo & Velocidad & Capacidad de análisis & Limitaciones \\
		\hline
		FASTA & 1 & 2 & 3 & 4 \\
		\hline 
		CAP3 & 1 & 2 & 3 & 4 \\
		\hline
		Bowtie & 1 & 2 & 3 & 4 \\
		\hline
	\end{tabular}

	\textbf{Limitaciones:}
	\begin{itemize}
		\item [FASTA]:
		\item [CAP3]:
		\item [Bowtie]:
	\end{itemize}

	\item \textbf{Scripting}:
	\begin{enumerate}
		\item Escribir un script en Perl (Python) donde, utilizando el módulo de BioPerl (BioPython), se pueda bajar una secuencia FASTA de genbank: EU856392
		\item Abra e imprima el archivo fasta
		\item Traduzca la secuencia a proteína
		\item Realice un blastp contra la base de datos de proteínas
		\item Imprima el Resultado del Blast
	\end{enumerate}


\end{enumerate}

%\chapter{Un Capitulo}
%\label{refcapitulo}
%aca va la info.\cite{sergiomonsalve}

\bibliographystyle{ieeetr}
\bibliography{Referencias.bib}

%\newpage
%\appendix
%\chapter{Repositorio del código}
%\input{caps/Apendix}
%Aca va el apendice

\end{document}
